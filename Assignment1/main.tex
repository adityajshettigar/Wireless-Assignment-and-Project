\documentclass[a4paper,fleqn]{cas-dc}
\usepackage[numbers]{natbib}

\usepackage{mathtools}
\usepackage{booktabs}

\usepackage{flushend}


\usepackage{graphicx}
\usepackage{algorithm}
\usepackage{algorithmic}
\usepackage{amsmath}
\usepackage{amssymb}  
\usepackage{tikz}
\usepackage{subcaption}

\numberwithin{equation}{section}

\begin{document}
\let\WriteBookmarks\relax
\def\floatpagepagefraction{1}
\def\textpagefraction{.001}

\shorttitle{Scalable Security Architectures for Massive IoT and Wireless Networks}
\shortauthors{Shettigar A. J. et~al.}

\title [mode = title]{Scalable Security Architectures for Massive IoT and Wireless Networks}                      


\author[1]{Shettigar A.J.}[orcid= 0009-0009-4257-2756]
\ead{aditya.23mic7018@vitapstudent.ac.in}
\address[1]{School of Computer Science and Engineering, VIT-AP University, Amaravati. Andhra Pradesh, India}



\cortext[cor1]{Corresponding author}

\begin{abstract}
As more Internet of Things (IoT) devices connect to our networks, the risk of advanced cyberattacks grows. Moving into the era of 6G and "Massive IoT," traditional security methods simply can't keep up. Small edge devices don't have the battery or processing power to handle heavy encryption, and relying on a central server causes delays and creates a single point of failure—if that server is breached, the whole network is at risk.
To solve this, we need security that is lightweight, decentralized, and easy to scale. This paper introduces a new framework combining Software-Defined Networking (SDN), Distributed Ledger Technology (DLT), and Zero-Trust principles (where no device is trusted by default). By using AI to verify devices and advanced cryptography built to resist future quantum computers, our design ensures that massive wireless networks stay fast, private, and highly secure, even in unpredictable environments.
\end{abstract}

\begin{keywords}
Massive IoT (MIoT) \sep 
Zero-Trust Architecture \sep
6G Wireless Networks \sep
Lightweight Authentication \sep
Distributed Ledger Technology (DLT) \sep
Software-Defined Networking (SDN) \sep
Post-Quantum Cryptography \sep
Network Slicing \sep
\end{keywords}

\maketitle

\section{Introduction}
The introduction of Massive Internet of Things (MIoT) into critical sectors, such as smart cities and healthcare, has shifted the primary engineering objective from basic connectivity to extreme security \cite{11358823, healthcare}. In these environments, unauthorized access or data breaches can trigger physical and economic consequences \cite{Chatterjee_2023_PUF}. Unlike consumer-grade systems, critical applications demand secure performance, ensuring data packets are protected against attacks like radio jamming \cite{namvar2016jamminginternetthingsgametheoretic} while meeting strict time bounds for grant-free access \cite{Liu2020AnalyzingGA}. Achieving this is increasingly difficult due to the massive scale of devices and their embedded hardware limitations. Consequently, traditional security protocols are often insufficient for capturing the dynamic threat models of modern wireless networks, necessitating a fundamental redesign of architectures to guarantee operational safety.



In order to address these challenges, this paper proposes a cross-layer design framework aimed at maximizing network security and data integrity in large-scale environments. We first optimize access control using Software-Defined Networking (SDN) to ensure robust policy management even during distributed attacks \cite{softwareiot, Al_2023_SDN_Security}. To further mitigate centralized vulnerabilities, we integrate Distributed Ledger Technologies (DLT) that dynamically secure the communication environment across thousands of nodes \cite{blockchain, Yang_2024_DAG}. At the protocol layer, we examine post-quantum cryptographic schemes that allow massive machine-type communications to remain secure against future computing threats \cite{post_quantummoura, ztiot}. Finally, to eliminate outdated trust assumptions, the architecture incorporates decentralized Zero-Trust mechanisms, ensuring that the network remains self-protecting even if individual edge components are compromised.

\subsection{Research Objectives}

The primary objective focuses on designing lightweight authentication protocols for resource-constrained devices using Physical Unclonable Functions (PUFs) and advanced neural networks. This approach optimizes the physical layer authentication using Convolutional Neural Networks (CNNs) \cite{QIU20252025} to maximize security while minimizing computational overhead in smart industrial environments \cite{Sazib_2024_Lightweight}.

The second objective is to develop and analyze 5G and 6G cellular security protocols, including network slicing and grant-free access mechanisms. The goal is to guarantee the isolation of secure data slices \cite{s25133940} and implement Authentication and Acknowledgement (AA) approaches \cite{V2024101669} without compromising the performance of massive sensor deployments \cite{Liu_2023_FormalAnalysis}.

The third objective is the integration of Software-Defined Networking (SDN) and Zero-Trust frameworks into the wireless IoT architecture to enhance access control and isolate network threats \cite{softwareiot}. By dynamically configuring security policies at the software level, the system aims to apply quantum-resistant Zero-Trust principles \cite{ztiot} to ensure secure connections for large-scale IoT networks \cite{Al_2023_SDN_Security}.

The fourth objective is to design a decentralized security architecture based on Distributed Ledger Technologies (DLT) and Directed Acyclic Graphs (DAG) to enhance data integrity \cite{Yang_2024_DAG}. This aims to eliminate single points of failure associated with centralized authorities by distributing trust and cryptographic functions across industrial IoT networks \cite{blockchain}.

The final objective is to define and implement forward-looking security measures, such as quantum-empowered learning and semantic-aware security protocols \cite{Zhou_2025_SemanticSec}. These advanced models validate the proposed architectures against future threats, ensuring they meet the rapid deployment demands of next-gen critical IoT applications \cite{JAVEED2024577, s19112484}.

\section{Literature Survey}

Existing literature on IoT security predominantly examines individual cryptographic protocols in isolation rather than adopting a cross-layer strategy. Early attempts to secure IoT relied heavily on traditional hash-based methods; however, these approaches struggle to adapt to the limited power constraints of healthcare devices and smart cities \cite{Sazib_2024_Lightweight, Chatterjee_2023_PUF}. To address these limitations, recent research has shifted toward lightweight authentication schemes that inherently maximize privacy and energy efficiency \cite{healthcare}. Furthermore, modern systems are integrating AI-driven physical layer authentication using CNNs to better handle spoofing attacks \cite{QIU20252025}, while also addressing fundamental wireless vulnerabilities such as malicious jamming from a game-theoretic perspective \cite{namvar2016jamminginternetthingsgametheoretic}.

At the network layers, the evolution toward 5G and 6G presents a significant security bottleneck, particularly when managing massive connectivity and Ultra-Reliable Low-Latency Communication (URLLC). Recent studies propose formal security analyses of 5G-AKA protocols and grant-free access techniques to balance security requirements without sacrificing rapid transmission \cite{Liu_2023_FormalAnalysis, Liu2020AnalyzingGA}. Similarly, network segmentation is being implemented to logically separate traffic, thereby improving isolation and security \cite{s25133940}. Although cutting shows promise, dynamically configuring these partitions to meet the strict security limits of 6G remains a complex challenge that requires sophisticated Authentication and Acknowledgment (AA) strategies \cite{V2024101669, 11358823}.

To manage these complex networks, Software-Defined Networking (SDN) is being used to centralize security policy enforcement and overcome rigid hardware limitations \cite{softwareiot}, though securing the SDN controllers themselves requires further exploration \cite{Al_2023_SDN_Security}. Security architectures face similar trade-offs; standard deployments are vulnerable to rapid exploitation, requiring architectures that can be rapidly deployed with built-in data security \cite{s19112484}. Finally, traditional cryptographic metrics fail to capture the upcoming threat of quantum computing, highlighting the urgent need for a unified framework that integrates post-quantum cryptography \cite{post_quantummoura} and quantum-resistant Zero-Trust frameworks \cite{ztiot}.

Finally, the architectural focus is moving from centralized management to decentralized resilience. Centralized security creates single points of failure, prompting the adoption of Distributed Ledger Technology (DLT) and blockchains to distribute trust across industrial networks \cite{blockchain}. To support this in low-latency environments, DAG-based ledgers are being explored to process transactions faster than traditional blockchains \cite{Yang_2024_DAG}. However, validating these systems requires new approaches; traditional machine learning compromises data privacy, leading to the rise of quantum-empowered federated learning \cite{JAVEED2024577}. Consequently, there is an urgent need for semantic-aware security architectures that account for both the meaning of the transmitted data and the massive scale of 6G networks \cite{Zhou_2025_SemanticSec}.

\subsection{Table for Literature Survey}
Table \ref{tab:literature_summary} summarizes the key research contributions related to scalable security architectures in Massive IoT. It highlights the shift from traditional centralized security designs to dynamic, decentralized systems. Early approaches focused on simple lightweight authentication, but recent works utilize advanced techniques like Software-Defined Networking (SDN) for access control and Distributed Ledger Technology (DLT) for trust distribution. While these modern methods resolve critical issues like single points of failure and quantum vulnerability, they often introduce new challenges regarding computational complexity and dynamic management overhead. The table compares these proposed methods, the specific techniques employed, the problems they solve, and their existing limitations.

\begin{table*}[width=.9\textwidth,cols=5,pos=h]
\caption{Comparison of Scalable Security Architecture Approaches}
\label{tab:literature_summary}
\begin{tabular*}{\textwidth}{@{\extracolsep{\fill}} p{0.15\textwidth} p{0.2\textwidth} p{0.2\textwidth} p{0.2\textwidth} p{0.2\textwidth} }
\toprule
\textbf{Ref.} & \textbf{Method Proposed} & \textbf{Techniques Used} & \textbf{Issues Resolved} & \textbf{Limitation} \\
\midrule

\cite{Sazib_2024_Lightweight} & 
Lightweight Mutual Auth & 
Hash-based Protocols & 
High overhead in traditional security methods & 
Vulnerable to future quantum computing attacks. \\
\addlinespace

\cite{Al_2023_SDN_Security} & 
SDN-Based Zero Trust & 
Software-Defined Networking & 
Rigid access control in large-scale networks & 
SDN controller becomes a high-value attack target. \\
\addlinespace

\cite{Yang_2024_DAG} & 
DAG-Based Ledger & 
Distributed Ledger Technology & 
Single point of failure in centralized servers & 
High memory requirements for individual edge nodes. \\
\addlinespace

\cite{JAVEED2024577} & 
Quantum Federated Learning & 
Federated Learning \& 6G & 
Data privacy leaks during centralized AI training & 
Complex synchronization across massive global nodes. \\
\addlinespace

\cite{s25133940} & 
5G Slicing Security & 
Network Slicing & 
Resource contention and lack of traffic isolation & 
Dynamic slice management overhead under high load. \\
\addlinespace

\cite{QIU20252025} & 
Physical Layer Authentication & 
Lightweight CNNs & 
Spoofing attacks in wireless communication channels & 
Training requires massive amounts of clean channel data. \\

\bottomrule
\end{tabular*}
\end{table*}

\subsection{Research Gaps}
Despite significant advancements in IoT security, several critical gaps remain in the existing literature. First, while lightweight authentication techniques have been developed for edge devices, they largely fail to incorporate post-quantum cryptography, leaving networks vulnerable to future threats \cite{Sazib_2024_Lightweight, post_quantummoura}. Second, at the architectural layer, SDN provides excellent flexibility for Zero-Trust policies \cite{Al_2023_SDN_Security}, but centralized controllers remain a vulnerability bottleneck. Third, although Distributed Ledgers (DLTs) show promise for decentralizing trust, executing them efficiently on battery-powered nodes without massive latency spikes remains largely unexplored \cite{blockchain, Yang_2024_DAG}. Furthermore, securing grant-free access for URLLC traffic is difficult when combined with the complex security handshakes of 5G-AKA \cite{Liu2020AnalyzingGA, Liu_2023_FormalAnalysis}. Finally, merging 6G semantic-aware security \cite{Zhou_2025_SemanticSec} with advanced federated learning models \cite{JAVEED2024577} lacks a unified testing framework for massive industrial deployments.

\subsection{Motivation}
The transition of IoT from simple sensor networks to massive, critical infrastructure—such as smart healthcare, autonomous transportation, and industrial automation—necessitates a fundamental shift in security design. In these Massive IoT (MIoT) domains, the cost of a cyberattack is measured not just in data loss, but in physical safety and operational collapse \cite{healthcare, Chatterjee_2023_PUF}. Current heavy cryptographic models are insufficient for edge devices plagued by low processing power and battery constraints. There is an urgent need for a system that does not merely react to breaches but proactively prevents them through decentralized, lightweight design. The motivation for this research stems from the requirement to bridge the gap between heavy, traditional IT security and the harsh reality of resource-constrained IoT deployments. By creating a unified framework that addresses lightweight authentication, SDN-based access control, and post-quantum resilience simultaneously, we aim to enable highly secure, next-generation wireless networks.

\subsection{Problem Statement}
The core problem addressed in this study is the inability of current security architectures to scale efficiently and protect massive IoT deployments against advanced cyber threats. Specifically, this research tackles four distinct sub-problems. First, traditional authentication protocols are too computationally heavy for edge devices, yet lightweight alternatives are vulnerable to spoofing and quantum attacks \cite{QIU20252025, post_quantummoura}. Second, reliance on centralized security servers creates vulnerability bottlenecks, where a single successful attack can compromise the entire network \cite{Al_2023_SDN_Security}. Third, the wireless medium is highly susceptible to physical layer attacks, such as jamming, which disrupt critical low-latency communications \cite{namvar2016jamminginternetthingsgametheoretic, V2024101669}. Fourth, maintaining data privacy while training network intelligence models is difficult without exposing sensitive information. This research seeks to formulate a comprehensive solution that optimizes protocol lightness, decentralizes trust using DAGs, and hardens the physical layer to ensure data integrity despite these adversarial conditions.

\section{Conclusion}

Scalable Security Architectures are essential for the safe deployment of the next generation of wireless Massive IoT. By shifting the design focus from heavy, centralized firewalls to lightweight, decentralized Zero-Trust frameworks, we can support critical applications where a security breach is not an option. Our research demonstrates that through a combination of Software-Defined Networking, Distributed Ledger Technology, and AI-driven physical layer authentication, wireless networks can achieve robust protection without sacrificing latency or battery life. 

Future work will involve integrating quantum-resistant cryptographic algorithms and 6G-enabled semantic-aware security features to further harden these networks against extreme cyber volatility. As networks scale to billions of connected devices, maintaining this balance between efficiency and unbreakable security will be the defining challenge of the next decade of telecommunications research.

\section*{Declarations}
\subsection*{Ethical Approval}
This manuscript reports studies that do not involve human participants, human data, human tissue, or animals.

\subsection*{Conflict of Interest}
The authors have no conflict of interest to declare that are relevant to the content of this article.

\subsection*{Authors' Contributions}
Shettigar A.J. contributed to the conceptualization, Formal analysis, drafting the original manuscript, and designing the experimental protocols. Shettigar was also responsible for Conceptualization, methodology, software implementation, and dataset curation. Shettigar contributed to conceptualization, supervision, and in-depth analysis of the experimental results. Shettigar handled the review and editing of the final draft, as well as the overall validation of the study results.


\subsection*{Funding}
The author did not receive financial support from any organization for the submitted work.

\subsection*{Availability of data and materials}
Data sharing is not applicable to this article, as no new data were created or analyzed in this study.

\subsection*{Acknowledgment}
We acknowledge that the hardware utilized in this research for evaluation is a part of the Intel IoT Center for Excellence, VIT-AP University. 

\bibliographystyle{cas-model2-names}
\bibliography{references}


\end{document}

